% !TeX spellcheck = en_US
%
%
% v 2.3  Feb 2019   Volker RW Schaa
%		# changes in the collaboration therefore updated file "jacow-collaboration.tex"
%		# all References with DOIs have their period/full stop before the DOI (after pp. or year)
%		# in the author/affiliation block all ZIP codes in square brackets removed as it was not 
%         understood as optional parameter and ZIP codes had bin put in brackets
%       # References to the current IPAC are changed to "IPAC'19, Melbourne, Australia"
%       # font for "url" style changed to "newtxtt" as it is easier to distinguish "O" and "0"
%
\documentclass[letter,
               %boxit,        % check whether paper is inside correct margins
               %titlepage,    % separate title page
               %refpage       % separate references
               biblatex,     % biblatex is used
               keeplastbox,   % flushend option: not to un-indent last line in References
               %nospread,     % flushend option: do not fill with whitespace to balance columns
               %hyphens,      % allow \url to hyphenate at "-" (hyphens)
               %xetex,        % use XeLaTeX to process the file
               %luatex,       % use LuaLaTeX to process the file
               ]{jacow}
%
% ONLY FOR \footnote in table/tabular
%
\usepackage{pdfpages,multirow,ragged2e} %
%
% CHANGE SEQUENCE OF GRAPHICS EXTENSION TO BE EMBEDDED
% ----------------------------------------------------
% test for XeTeX where the sequence is by default eps-> pdf, jpg, png, pdf, ...
%    and the JACoW template provides JACpic2v3.eps and JACpic2v3.jpg which
%    might generates errors, therefore PNG and JPG first
%
\makeatletter%
	\ifboolexpr{bool{xetex}}
	 {\renewcommand{\Gin@extensions}{.pdf,%
	                    .png,.jpg,.bmp,.pict,.tif,.psd,.mac,.sga,.tga,.gif,%
	                    .eps,.ps,%
	                    }}{}
\makeatother

% CHECK FOR XeTeX/LuaTeX BEFORE DEFINING AN INPUT ENCODING
% --------------------------------------------------------
%   utf8  is default for XeTeX/LuaTeX
%   utf8  in LaTeX only realises a small portion of codes
%
\ifboolexpr{bool{xetex} or bool{luatex}} % test for XeTeX/LuaTeX
 {}                                      % input encoding is utf8 by default
 {\usepackage[utf8]{inputenc}}           % switch to utf8

\usepackage[USenglish]{babel}


%
% if BibLaTeX is used
%
\ifboolexpr{bool{jacowbiblatex}}%
 {%
  \addbibresource{WEPHA044.bib}
  \addbibresource{biblatex-examples.bib}
 }{}
\listfiles

%%
%%   Lengths for the spaces in the title
%%   \setlength\titleblockstartskip{..}  %before title, default 3pt
%%   \setlength\titleblockmiddleskip{..} %between title + author, default 1em
%%   \setlength\titleblockendskip{..}    %afterauthor, default 1em

\begin{document}

\title{Upgrade of the Bunch Length and Bunch Charge Control Systems for the New SLAC Free Electron Laser}

\author{M. P. Donadio\thanks{marcio@slac.stanford.edu}, A. S. Fisher, L. Sapozhnikov, SLAC, Menlo Park, CA 94025, USA}
	
\maketitle


\begin{abstract}
In 2019 SLAC is building a new linear accelerator based on superconducting niobium cavities. The first one, now called the copper linac, could generate 120 electron bunches per second. The new one, called superconducting linac, will generate 1 million per second, bringing some challenges to many devices along with the accelerator. Most of them receive sensors and actuators in a VME-based Platform with its control running in software, with RTEMS as OS. This is feasible for 120 Hz, but not for 1 MHz. The new control hardware is ATCA-based Platform, that has carrier boards with FPGA connected to servers running Embedded real-time Linux OS, forming the High-Performance System (HPS). Instead of having all the new architecture installed at the accelerator and tested on the go, SLAC used the strategy of testing the systems in the copper linac, to have them ready to use in the superconducting linac in what was called the Mission Readiness Program. The Bunch Length System and the Bunch Charge System are examples of devices of this program. Both systems were tested in the copper linac at 120 Hz, with excellent results. The next step is to test them at the superconducting linac, at 1 MHz.
\end{abstract}


\section{Introduction}
LCLS has two bunch length monitors based on pyro detectors\cite{blen-pac07} and xxx bunch charge monitors based on Toroids, the two types of devices that were in the scope of this upgrade. Beyond these devices, LCLS uses gap diode to measure the bunch length\cite{injector-commissioning-2007} and Faraday cup and BPMs to measure the bunch charge [reference].

\subsection{Pyro Detector Bunch Length Structure}
LCLS has two bunch compressors (BC). A hole mirror is placed after the last bending magnet of each BC, generating diffraction radiation (DR) from the edge radiation (ER) (see Fig. \ref{fig:blen_struct1}). The DR is, then, reflected, split, and filtered before getting to two pyro detectors (see Fig. \ref{fig:blen_struct2}). \cite{blen-pac07}

By using specific optical filters it is possible to measure difference in the intensity of the signals at the pyro detector output for different lengths of the bunch. Comparing signal intensity for different bunches it is possible to know that one bunch is larger or shorter relative to the other. Calculating the absolute length is only possible when the charge of the beam is used for normalization. [reference]

\begin{figure}[!htb]
   \centering
   \includegraphics*[width=\columnwidth]{BunchLengthStruct_1}
   \caption{ER from bending magnet generating DR through a hole mirror. Reproduced from \cite{blen-pac07} with permission from the authors.}
   \label{fig:blen_struct1}
\end{figure}

\begin{figure}[!htb]
   \centering
   \includegraphics*[width=\columnwidth]{BunchLengthStruct_2}
   \caption{Optical structure of the bunch length system showing how the DR is reflected, split, and filtered before getting to the pyro detectors. Reproduced from \cite{blen-pac07} with permission from the authors.}
   \label{fig:blen_struct2}
\end{figure}

\subsection{Toroid Bunch Charge Structure}
A toroid is placed at the accelerator in a way that the electron beam passes through its hole, inducting a current in the turns. This current is conditioned as it can have enough width for the temporal resolution of the digitizer. The digitized signal is integrated resulting in the bunch charge. [reference]

[figure showing structure]

\section{Bunch Length System}
The signal of the pyro detectors is digitalized, processed by an equation that results in the length of the bunch, and timestamped to log a correspondence between the result and the electron bunch among all the accelerator systems. In the next two sections we will show the architecture of the previous VME-based system and how it was upgraded to an ATCA-based architecture.

\subsection{VME-based architecture}
Figure \ref{fig:blen_vme} shows the high-level process of the original bunch length system.

\begin{figure}[!htb]
   \centering
   \includegraphics*[width=\columnwidth]{BLEN_VME_Process}
   \caption{The original (before the upgrade) bunch length system process.}
   \label{fig:blen_vme}
\end{figure}

\begin{enumerate}
   \item The timing system sends a packet containing the timing pattern with event codes and the timestamp. The EVR in the VME crate receives this data. The IOC running in VME waits for a specific event code. When it arrives, the timestamp is registered.
   \item Using the IP 445 module, the IOC running in VME sends a trigger to the digitizer, telling it to start collecting data from the pyro detector.
   \item The IOC running in the VME tells to the IOC running in the Digitizer that it is waiting for data as soon as the Digitizer finishes its digitalization. The VME IOC also sends a timestamp that will be used by the Digitizer IOC as a data label.
   \item When the digitalization is complete, the Digitizer IOC sends an UDPComm packet to the VME IOC with the digitized data containing one raw waveform per each of the pyro detectors. The data is labeled with the timestamp previously sent by the VME IOC. 
   \item Parallel to the communication with the Digitizer IOC, the VME IOC receives the charge of the bunch (TMIT) sent by the BPM system, through an FCOM packet.
   \item The VME IOC confirms the timestamp from the EVR, the Digitizer, and the BPM system and, if they match, it calculates the bunch length.
   \item The result of the calculation is recorded in the many EPICS records that are part of the BSA system.
   \item The VME IOC sends an FCOM package to the Fast Feedback System, containing the calculated bunch length.
\end{enumerate}

As seen, all the process is held by software running in RTEMS OS. Since we are in a condition where the minimum distance between the bunches is 8.33 ms, it is feasible to let the software running in a real-time OS take care of all the processing. The new XFEL, though, will operate at maximum of 1 MHz frequency, which reduces time between bunches to \SI{1}{\micro\meter}. For this time range, another architecture is needed. [reference]

\subsection{ATCA-based architecture}
To work with high-processing speed, SLAC has chosen to use FPGAs. The architecture of choice to have the FPGAs on board is ATCA [reference].

A project called Mission Readiness was created to substitute the VME-based architecure with the ATCA, but still using the 120 Hz frequency. The goal was to verify if LCLS-I would still operate as before after the control hardware was changed. [reference]

Figure \ref{fig:blen_atca} shows how the new architecture was applied to the Bunch Length System of LCLS-I. Here we don't show the complete ATCA architecture for the sake of simplicity, but detailed description can be found in \cite{ryan-2016}\cite{atca-bpm-2017}.

\begin{figure}[!htb]
   \centering
   \includegraphics*[width=\columnwidth]{BLEN_ATCA_Process}
   \caption{The upgraded bunch length system process.}
   \label{fig:blen_atca}
\end{figure}


\section{Bunch Charge System}
bla bla bla.


\begin{table}[!hbt]
   \centering
   \caption{Margin Specifications}
   \begin{tabular}{lcc}
       \toprule
       \textbf{Margin} & \textbf{A4 Paper}                      & \textbf{US Letter Paper} \\
       \midrule
           Top         & \SI{37}{mm} (\SI{1.46}{in})            & \SI{0.75}{in} (\SI{19}{mm})        \\ %[3pt]
          Bottom       & \SI{19}{mm} (\SI{0.75}{in})            & \SI{0.75}{in} (\SI{19}{mm})        \\ %[3pt]
           Left        & \SI{20}{mm} (\SI{0.79}{in})            & \SI{0.79}{in} (\SI{20}{mm})        \\ %[3pt]
           Right       & \SI{20}{mm} (\SI{0.79}{in})            & \SI{1.02}{in} (\SI{26}{mm})        \\
       \bottomrule
   \end{tabular}
   \label{tab:margins}
\end{table}

\subsection{subsection 2.2}
bla bla bla.

\begin{figure}[!htb]
   \centering
   \includegraphics*[width=.7\columnwidth]{JACpic_mc}
   \caption{Layout of papers.}
   \label{fig:paper_layout}
\end{figure}

\begin{figure*}[!tbh]
    \centering
    \includegraphics*[width=\textwidth]{JACpic2}

    \caption{Example of a full-width figure showing the JACoW Team at their annual
    	     meeting in December 2018. This figure has a multi-line caption that has to be
    	     justified rather than centred.} %US
    \label{fig:jacow_team}
\end{figure*}

\subsection{subsection 2.3}

bla bla bla.

\subsection{subsection 2.4}

bla bla bla.

\subsection{subsection 2.5}

bla bla bla.

\subsubsection{subsubsection 2.5.1} bla bla bla.

\subsection{subsection 2.6}

bla bla bla.

\subsection{subsection 2.7}

bla bla bla.

\begin{equation}\label{eq:label}
    C_B=\frac{q^3}{3\epsilon_{0} mc}=\SI{3.54}{\micro eV/T}
\end{equation}

\subsection{subsection 2.8}
	
bla bla bla.

\subsection{subsection 2.9}
bla bla bla.

  \begin{table}[h!b]
	\setlength\tabcolsep{3.5pt}
	\centering
	\caption{Summary of Styles}
	\label{tab:styles}
	\begin{tabular}{llcc}
		\toprule
		\textbf{Style} & \textbf{Font}               & \textbf{Space}  & \textbf{Space} \\
		&                             & \textbf{Before} & \textbf{After} \\
		\midrule
		\textbf{PAPER}  & \SI{14}{pt}                 & \SI{0}{pt}      & \SI{3}{pt}  \\
		\textbf{TITLE}  & \textbf{UPPERCASE}          &                 &      \\
		& \textbf{EXCEPT FOR}         &                 &      \\
		& \textbf{REQUIRED lowercase} &                 &      \\
		& \textbf{letters}            &                 &      \\
		& \textbf{Bold}               &                 &      \\[5pt]
		%\midrule
		Author list  & \SI{12}{pt}                 & \SI{9}{pt}      & \SI{12}{pt} \\
		& UPPER- and lowercase        &                 &      \\[5pt]
		%\midrule
		\textit{Abstract} & \SI{12}{pt}                 & \SI{0}{pt}      & \SI{3}{pt} \\
		\textit{Title}  & \textit{Initial Caps}       &                 &      \\
		& \textit{Italic}             &                 &      \\[5pt]
		%\midrule
		\textbf{Section}  & \SI{12}{pt}                 & \SI{9}{pt}      & \SI{3}{pt}  \\
		\textbf{Heading}  & \textbf{UPPERCASE}          &                 &      \\
		& \textbf{bold}               &                 &      \\[5pt]
		%\midrule
		\textit{Subsection} & \SI{12}{pt}                 & \SI{6}{pt}      & \SI{3}{pt}  \\
		\textit{Heading}
                             & \textit{Initial Caps}       &                 &      \\
		& \textit{Italic}             &                 &      \\[5pt]
		%\midrule
		\textbf{Third-level} & \SI{10}{pt}                 & \SI{6}{pt}      & \SI{0}{pt}  \\
		\textbf{Heading}     
                            & \textbf{Initial Caps}       &                 &      \\
		& \textbf{Bold}               &                 &      \\[5pt]
		%\midrule
		Figure        & \SI{10}{pt}                 & \SI{3}{pt}      & $\ge$\SI{3}{pt}  \\
		Captions      &                             &                 &      \\[5pt]
		%\midrule
		Table         & \SI{10}{pt}                 & $\ge$\SI{3}{pt} & \SI{3}{pt}  \\
		Captions	  &                             &                 &      \\[5pt]
		%\midrule
		Equations     & \SI{10}{pt} base font       & $\ge$\SI{6}{pt}     & $\ge$\SI{6}{pt} \\[5pt]
		%\midrule
		References      & \SI{9}{pt}				& \SI{0}{pt}      & \SI{3}{pt} \\
        when $\le9$ 	& \verb|\bibliography{9}|	&                 &  \\[5pt]
        Refs. $\ge10$ 	& \SI{9}{pt}				& \SI{0}{pt}      & \SI{3}{pt}  \\
                		& \verb|\bibliography{99}|	&    &    \\
		\bottomrule   %\SI{0.25}{in}
	\end{tabular}
\end{table}

\subsection{subsection 2.10}

bla bla bla.

\subsection{subsection 2.11}

bla bla bla. 

\section{section 3}

bla bla bla.

\section{section 4}

bla bla bla.

\section{section 5}

bla bla bla.

\section{Results and Discussion}

bla bla bla.


\section{CONCLUSION}

bla bla bla.

\section{ACKNOWLEDGMENTS}

bla bla bla.


%
% only for "biblatex"
%
\ifboolexpr{bool{jacowbiblatex}}%
	{\printbibliography}%
	{%
	% "biblatex" is not used, go the "manual" way
	
	%\begin{thebibliography}{99}   % Use for  10-99  references
	\begin{thebibliography}{9} % Use for 1-9 references
	
	\bibitem{jacow-help}
		JACoW,
		\url{http://www.jacow.org}
	
	\bibitem{IEEE}
		\textit{IEEE Editorial Style Manual},
		IEEE Periodicals, Piscataway,
		NJ, USA, Oct. 2014, pp. 34--52.

	\bibitem{journal-abbreviations}
	\url{https://woodward.library.ubc.ca/researchhelp/journal-abbreviations/}

	\end{thebibliography}
} % end \ifboolexpr

\end{document}
