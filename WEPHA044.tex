% !TeX spellcheck = en_US
%
%
% v 2.3 Feb 2019  Volker RW Schaa
%		# changes in the collaboration therefore updated file "jacow-collaboration.tex"
%		# all References with DOIs have their period/full stop before the DOI (after pp. or year)
%		# in the author/affiliation block all ZIP codes in square brackets removed as it was not 
%     understood as optional parameter and ZIP codes had bin put in brackets
%    # References to the current IPAC are changed to "IPAC'19, Melbourne, Australia"
%    # font for "url" style changed to "newtxtt" as it is easier to distinguish "O" and "0"
%
\documentclass[letter,
        %boxit,    % check whether paper is inside correct margins
        %titlepage,  % separate title page
        %refpage    % separate references
        biblatex,   % biblatex is used
        keeplastbox,  % flushend option: not to un-indent last line in References
        %nospread,   % flushend option: do not fill with whitespace to balance columns
        %hyphens,   % allow \url to hyphenate at "-" (hyphens)
        %xetex,    % use XeLaTeX to process the file
        %luatex,    % use LuaLaTeX to process the file
        ]{jacow}
%
% ONLY FOR \footnote in table/tabular
%
\usepackage{pdfpages,multirow,ragged2e} %
%
% CHANGE SEQUENCE OF GRAPHICS EXTENSION TO BE EMBEDDED
% ----------------------------------------------------
% test for XeTeX where the sequence is by default eps-> pdf, jpg, png, pdf, ...
%  and the JACoW template provides JACpic2v3.eps and JACpic2v3.jpg which
%  might generates errors, therefore PNG and JPG first
%
\makeatletter%
	\ifboolexpr{bool{xetex}}
	 {\renewcommand{\Gin@extensions}{.pdf,%
	          .png,.jpg,.bmp,.pict,.tif,.psd,.mac,.sga,.tga,.gif,%
	          .eps,.ps,%
	          }}{}
\makeatother

% CHECK FOR XeTeX/LuaTeX BEFORE DEFINING AN INPUT ENCODING
% --------------------------------------------------------
%  utf8 is default for XeTeX/LuaTeX
%  utf8 in LaTeX only realises a small portion of codes
%
\ifboolexpr{bool{xetex} or bool{luatex}} % test for XeTeX/LuaTeX
 {}                   % input encoding is utf8 by default
 {\usepackage[utf8]{inputenc}}      % switch to utf8

\usepackage[USenglish]{babel}


%
% if BibLaTeX is used
%
\ifboolexpr{bool{jacowbiblatex}}%
 {%
 \addbibresource{WEPHA044.bib}
 \addbibresource{biblatex-examples.bib}
 }{}
\listfiles

%%
%%  Lengths for the spaces in the title
%%  \setlength\titleblockstartskip{..} %before title, default 3pt
%%  \setlength\titleblockmiddleskip{..} %between title + author, default 1em
%%  \setlength\titleblockendskip{..}  %afterauthor, default 1em

\begin{document}

\title{Upgrade of the Bunch Length and Bunch Charge Control Systems for the New SLAC Free Electron Laser}

\author{M. P. Donadio\thanks{marcio@slac.stanford.edu}, A. S. Fisher, L. Sapozhnikov, SLAC, Menlo Park, CA 94025, USA}
	
\maketitle


\begin{abstract}
In 2019 SLAC is building a new linear accelerator based on superconducting niobium cavities. The first one, now called the copper linac, could generate 120 electron bunches per second. The new one, called superconducting linac, will generate 1 million per second, bringing some challenges to many devices along with the accelerator. Most of them receive sensors and actuators in a VME-based Platform with its control running in software, with RTEMS as OS. This is feasible for 120 Hz, but not for 1 MHz. The new control hardware is ATCA-based Platform, that has carrier boards with FPGA connected to servers running Embedded real-time Linux OS, forming the High-Performance System (HPS). Instead of having all the new architecture installed at the accelerator and tested on the go, SLAC used the strategy of testing the systems in the copper linac, to have them ready to use in the superconducting linac in what was called the Mission Readiness Program. The Bunch Length System and the Bunch Charge System are examples of devices of this program. Both systems were tested in the copper linac at 120 Hz, with excellent results. The next step is to test them at the superconducting linac, at 1 MHz.
\end{abstract}


\section{Introduction}
LCLS has two bunch length monitors (BLEN) based on pyro detectors \cite{blen-pac07} and 13 bunch charge monitors based on Toroids \cite{slac-toroids}, the two types of devices that are in the scope of this upgrade. Beyond these devices, LCLS uses gap diode to measure the bunch length \cite{injector-commissioning-2007} and, to measure the bunch charge, Faraday cups \cite{slac-toroids} and BPMs \cite{beam-measurement}.

\subsection{Pyro Detector Bunch Length Structure}
LCLS has two bunch compressors (BC). A hole mirror is placed after the last bending magnet of each BC, generating diffraction radiation (DR) from the edge radiation (ER) (see Fig. \ref{fig:blen_struct1}). The DR is, then, reflected, split, and filtered before getting to two pyro detectors (see Fig. \ref{fig:blen_struct2}). \cite{blen-pac07}

By using specific optical filters it is possible to measure the difference in the intensity of the signals at the pyro detector output for different lengths of the bunch. Comparing signal intensity for different bunches it is possible to know that one bunch is larger or shorter relative to the other. Calculating the absolute length is only possible when the charge of the beam is used for normalization. [reference]

\begin{figure}[!htb]
  \centering
  \includegraphics*[width=\columnwidth]{BunchLengthStruct_1}
  \caption{ER from bending magnet generating DR through a hole mirror. Reproduced from \cite{blen-pac07} with permission from the authors.}
  \label{fig:blen_struct1}
\end{figure}

\begin{figure}[!htb]
  \centering
  \includegraphics*[width=\columnwidth]{BunchLengthStruct_2}
  \caption{Optical structure of the bunch length system showing how the DR is reflected, split, and filtered before getting to the pyro detectors. Reproduced from \cite{blen-pac07} with permission from the authors.}
  \label{fig:blen_struct2}
\end{figure}

\subsection{Toroid Bunch Charge Monitor (BCM) Structure}
A toroid is placed at the accelerator in a way that the electron beam passes through its hole, inducing a current in the turns. This current is conditioned and digitized. By measuring the peak of the conditioned signal, the system can calculate the bunch charge. For an explanation of the theory, please refer to \cite{slac-pub-398}.

\subsection{The Mission Readiness Program}
SLAC Mission Readiness Program goal is to implement improvements in the present to guarantee that the laboratory will comply with its mission in the future and maintain it in the long-term \cite{doe-MR}. The upgrade of the control system for LCLS is part of this broader program.

As the description of the previous Bunch Length and Bunch Charge Systems will show in the next sections, before the LCLS upgrade all the process was held by software running in RTEMS OS. Since we were in a condition where the minimum distance between the bunches was 8.33 ms, it was feasible to let the software running in a real-time OS take care of all the processing. The new XFEL, though, will operate at a maximum of 1 MHz frequency, which reduces the time between bunches to \SI{1}{\micro\meter}. For this time range, another architecture is needed. [reference]

To work with high-processing speed, SLAC has chosen to use FPGAs. The architecture of choice to have the FPGAs on board was ATCA [reference].

Within the Mission Readiness Program there was the substitution of the VME-based architecture with the ATCA, still using the 120 Hz frequency. The goal was to verify if the Copper Linac would still operate as before after the control hardware was changed.

\section{Bunch Length System}
The signal of the pyro detectors is digitized, processed by an equation that results in the length of the bunch, and timestamped to log a correspondence between the result and the electron bunch among all the accelerator systems. In the next two sections, we will show the architecture of the previous VME-based system and how it was upgraded to an ATCA-based architecture.

\subsection{VME-based architecture}
Figure \ref{fig:blen_vme} shows the high-level process of the original bunch length system.

\begin{figure}[!htb]
  \centering
  \includegraphics*[width=.8\columnwidth]{BLEN_VME_Process}
  \caption{The original (before the upgrade) bunch length system process.}
  \label{fig:blen_vme}
\end{figure}

\begin{enumerate}
  \item The timing system sends a packet containing the timing pattern with event codes and the timestamp. The EVR in the VME crate receives this data. The IOC running in VME waits for a specific event code. When it arrives, the timestamp is registered.
  \item Using the IP 445 module, the IOC running in VME sends a trigger to the digitizer, telling it to start collecting data from the pyro detector.
  \item The IOC running in the VME tells to the IOC running in the Digitizer that it is waiting for data as soon as the Digitizer finishes its digitalization. The VME IOC also sends a timestamp that will be used by the Digitizer IOC as a data label.
  \item When the digitalization is complete, the Digitizer IOC sends an UDPComm packet to the VME IOC with the digitized data containing one raw waveform per each of the pyro detectors. The data is labeled with the timestamp previously sent by the VME IOC. 
  \item Parallel to the communication with the Digitizer IOC, the VME IOC receives the charge of the bunch (TMIT) sent by the BPM system, through an FCOM packet.
  \item The VME IOC confirms the timestamp from the EVR, the Digitizer, and the BPM system and, if they match, it calculates the bunch length.
  \item The result of the calculation is recorded in the many EPICS records that are part of the BSA system.
  \item The IOC sends an FCOM package to the Fast Feedback System, containing the calculated bunch length.
\end{enumerate}

\subsection{ATCA-based architecture}
Figure \ref{fig:blen_atca} shows how the new architecture was applied to the Bunch Length System of the Copper Linac. Here we don't show the complete ATCA architecture for the sake of simplicity, but the detailed description can be found in \cite{ryan-2016, atca-bpm-2017}.

\begin{figure}[!htb]
  \centering
  \includegraphics*[width=.8\columnwidth]{BLEN_ATCA_Process}
  \caption{The upgraded bunch length system process.}
  \label{fig:blen_atca}
\end{figure}

\begin{enumerate}
  \item Similarly to the VME system, the timing system sends a packet containing the timing pattern with event codes and the timestamp, but now, the ATCA RTM receives this information and send it to the Carrier Board.
  \item Now we have the FPGA firmware working with triggering the digitizer and waiting for the array of data to arrive.
  \item The same as in the VME system, the IOC receives the charge of the bunch (TMIT) sent by the BPM system, through an FCOM packet.
  \item The IOC sends the TMIT and timestamp to the FPGA firmware as it is an input for the bunch length calculation.
  \item The firmware calculates the bunch length.
  \item The result of the calculation is sent to the CPU.
  \item The result of the calculation is recorded in the many EPICS records that are part of the BSA system.
  \item The VME IOC sends an FCOM package to the Fast Feedback System, containing the calculated bunch length.
\end{enumerate}

The steps related to external systems (1, 3, 7, and 8) were kept the same as these systems wouldn't be changed.

\section{Bunch Charge System}
The conditioned signal of the toroid is digitized, processed by an equation that results in the charge of the bunch, and timestamped to log a correspondence between the result and the electron bunch among all the accelerator systems. In the next two sections, we will show the architecture of the previous VME-based system and how it was upgraded to an ATCA-based architecture.

\subsection{VME-based architecture}
Figure \ref{fig:bcm_vme} shows the high-level process of the original bunch charge system.

\begin{figure}[!htb]
  \centering
  \includegraphics*[width=.8\columnwidth]{BCM_VME_Process}
  \caption{The original (before the upgrade) bunch charge system process.}
  \label{fig:bcm_vme}
\end{figure}

\begin{enumerate}
  \item The signal conditioner that receives the short current pulse from multiple toroids is constantly comparing the signals between them and, when a bad condition is measured, it changes the digital signals connected to the Machine Protection System (MPS).
  \item The timing system sends a packet containing the timing pattern with event codes and the timestamp. The EVR in the VME crate receives this data.
  \item The IP 300 module digitizes the conditioned signal and the IOC reads its value.
  \item The IOC calculates the bunch charge.
  \item The IOC takes the bunch charge and the timestamp read in step 2 and stores it in BSA.
\end{enumerate}

\subsection{ATCA-based architecture}
Figure \ref{fig:bcm_atca} shows how the new architecture was applied to the Bunch Charge System of the Copper Linac. Here we don't show the complete ATCA architecture for the sake of simplicity, but the detailed description can be found in \cite{ryan-2016, atca-bpm-2017}.

\begin{figure}[!htb]
  \centering
  \includegraphics*[width=.8\columnwidth]{BCM_ATCA_Process}
  \caption{The upgraded bunch charge system process.}
  \label{fig:bcm_atca}
\end{figure}

\begin{enumerate}
  \item Similarly to the VME system, the timing system sends a packet containing the timing pattern with event codes and the timestamp, but now, the ATCA RTM receives this information and send it to the Carrier Board.
  \item Now we have the FPGA firmware working with triggering the digitizer and waiting for the array of data to arrive.
  \item The firmware calculates the bunch charge.
  \item One Carrier Board is connected to many toroids. The firmware calculates the charge based on all of them and sets RTM digital signals according to the comparison of the results. These digital signals are received by the MPS, that will react if a bad condition raises.
  \item The result of the calculation is sent to the CPU.
  \item The result of the calculation is recorded in the many EPICS records that are part of the BSA system.
\end{enumerate}

The same way as we've seen for the BLEN, the interfaces to the external systems were not changed.

\section{Results and Discussion}
\subsection{Bunch Length}
As seen in Fig. \ref{fig:blen_struct2}, the BLEN system uses two pyro detectors. To test the new system, one of them was connected to the digitizer in the ATCA. The accelerator was run with both systems operating at the same time. A positive test result should provide the same scalar number for the BLEN at the same beam pulse in both systems.

Both pyro detector BLEN systems, one in sector 21, after Bunch Compressor (BC) 1, and one in sector 24 after BC 2 were tested.

The test was performed for many different lengths. A correlation of 1 should be observed when we plot the measurement of one system against the other. Figure \ref{fig:blen_result} shows the scattering with the VME system in the x-axis and the ATCA system in the y-axis. The linear regression result shows that the correlation is close to 1, does confirming that the new system provides the expected results.

\begin{figure}[!htb]
  \centering
  \includegraphics*[width=\columnwidth]{blen_result}
  \caption{Measured bunch length correlation between the ATCA (Mission Readiness) and the VME (Current) system in sector 24.}
  \label{fig:blen_result}
\end{figure}

During a few tests, the Fast-Feedback system received data from the ATCA system and the accelerator could be operated normally, as before.

The filters were moved and we could check the reading of the limit switches according to what was expected.

\subsection{Bunch Charge}
The toroids in sector 24 and 25 were tested, following the procedure below. We call Input Toroid as the toroid upstream in sector 24, closer to the electron gun. The Output Toroid is downstream in sector 25.

\begin{enumerate}
  \item The MPS system was configured to ignore alerts from the toroids being tested.
  \item Still using the VME system for both toroids, we measured the correlation between the Input and the Output toroids to make sure that it was 1.
  \item We physically switched the Input Toroid cable to the ATCA system and measured the correlation with the Output Toroid running the VME system. Calibrating the ATCA system parameters, we could get a 1 correlation.
  \item We switched back the Input Toroid cable to the VME system and made the opposite with the Output Toroid. Again, adjusting the ATCA system parameters we've got a 1 correlation.
  \item Finally, we put both toroids in the ATCA system and measured successfully a correlation of 1.
\end{enumerate}

Figure \ref{fig:bcm_result} show the correlation when the Input Toroid was switched to the ATCA-based system.

\begin{figure}[!htb]
  \centering
  \includegraphics*[width=\columnwidth]{bcm_result}
  \caption{Measured bunch charge correlation between the ATCA (Mission Readiness) and the VME (Current) system in sectors 24 and 25.}
  \label{fig:bcm_result}
\end{figure}

Once both toroids were running in the ATCA system, we connected the digital signals to the MPS system and activated it. With the accelerator running, no signal tripped the MPS. We forced both systems to generate alerts to the MPS and it reacted blocking the injector, as expected.

Tests couldn't be done in sector 21 because the cables were too short to reach the new rack and there was no time slot before the start of the long 1-year downtime to install new cables.

\section{CONCLUSION}
Given the successful test results, the ATCA-based Bunch Length system is ready to replace the VME-based system in both sectors 21 and 24.

The ATCA-based Bunch Charge system is ready in sector 24 but will require calibration for the 4 toroids in sector 21 and tests with the MPS before the replacement takes place.

The Mission Readiness Program goal was achieved for BLEN and BCM systems, as it was proven that the ATCA-based systems could make the Copper Linac operate correctly at 120 Hz together with all other sub-systems and operator interfaces.

The next step is to test the ATCA-based system with the new Superconducting Linac at 1 MHz.

\section{ACKNOWLEDGMENTS}
Many people contributed to the success of this project. The authors would like to acknowledge Bryce Jacobson, Carolina Bianchini, Charles Granieri, Chris Ford, Ernest Williams, Hugo Slepicka, Jeremy Mock, Jesus Stanescu, Kenneth Brobeck, Kristi Luchini, Kukhee Kim, Luciano Piccoli, Maria Urbina, Matt Gibbs, Richard Truong, Shawn Alverson, Sonya Hoobler, Thuy Vu, Till Straumann, and Timothy Maxwell.

We acknowledge Josef Frisch and Juhao Wu for authorizing the use of the figures explaining the Bunch Length structure.

%
% only for "biblatex"
%
\ifboolexpr{bool{jacowbiblatex}}%
	{\printbibliography}%
	{%
	% "biblatex" is not used, go the "manual" way
	
	%\begin{thebibliography}{99}  % Use for 10-99 references
	\begin{thebibliography}{9} % Use for 1-9 references
	
	\bibitem{jacow-help}
		JACoW,
		\url{http://www.jacow.org}
	
	\bibitem{IEEE}
		\textit{IEEE Editorial Style Manual},
		IEEE Periodicals, Piscataway,
		NJ, USA, Oct. 2014, pp. 34--52.

	\bibitem{journal-abbreviations}
	\url{https://woodward.library.ubc.ca/researchhelp/journal-abbreviations/}

	\end{thebibliography}
} % end \ifboolexpr

\end{document}
